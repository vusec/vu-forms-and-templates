\subsection{Reproducibility}

In this mandatory section, describe the procedures to setup the environment for
evaluating or reusing your artifact(s).
%
The descriptions should be targeted at being accessible to novice users.

\paraib{Accessing the artifact(s)}
%
Describe how a reader can access your artifact(s).
%
If they are publicly available in a repository, provide the URL and clear
instructions (targeted at a novice user) on how to download an exact replica of
the artifact described in this appendix.
%
Suppose that the artifact is private, provide guidelines on how a reader could
approach relevant people (i.e., the author of this thesis or supervisors
involved in this project) to obtain access.

\paraib{Installation and configuration}
%
Describe how a (novice) user can install the artifact on their system or
platform.
%
Highlight the various hardware and software dependencies, and, explain in
detail, wherever applicable, how the user could install or configure these
dependencies.

\paraib{Experiment Workflow}
%
Provide a high-level overview of the (experiment) workflow, shedding light on
how you implemented it, how users can set it up and invoke it, and (optionally)
how they can customize or extend it.

\paraib{Evaluation and Expected Result}
%
Enumerate the key claims of your thesis and describe which experiment and result
supports that claim.
%
Explain, in detail, all the steps to reproduce each of these key results.
%
Where applicable, explain the maximum variation that is permissible in these
results.

\paraib{Experiment Customization}
%
If possible, and relevant, describe how to customize your workflow (e.g., to use
a different data set, benchmark, model, application, and environment).
%
If customizations are irrelevant to the artifact(s) produced in this thesis,
please remove this paragraph.


%%% Local Variables:
%%% mode: latex
%%% TeX-master: "../thesis"
%%% End:
