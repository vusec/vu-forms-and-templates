\subsection{Checklist}

This mandatory section provides a quick overview of the requirements for a
reader to evaluate or reuse the artifact(s) produced as part of your thesis.
%
Please approach your advisor if you have questions such as what constitutes an
artifact, how to make them accessible, and what level of detail to provide to
allow reuse of such artifacts.
%
We have provided a extensive checklist of items that are applicable to a broad
range of projects.
%
Remove items that are not relevant to your artifact(s); otherwise, simply
replace the description of each item, with details relevant to your artifact.
%
Leave the formatting intact.

\chk{Algorithm}:
%
If you are presenting a new algorithm, succinctly describe it along with 1-2 key
insights.

\chk{Benchmark}:
%
Describe any benchmarks (e.g., PARSEC, LINPACK, and SPEC) that you used in your
thesis and their sizes.
%
Explain how to access them.
%
If the benchmark is private, indicate whether it has a public analog.

\chk{Compilation}:
%
If your artifact requires a specific compiler, mention which compiler, what
version, and whether it is public.

\chk{Transformations}:
%
If your artifact uses a program transformation tool (source-to-source,
binary-to-binary, compiler pass, etc.), mention the tool, version, and whether
it is public.

\chk{Binary}:
%
Indicate whether your artifact(s) includes binaries, and, if it does, mention
their versions and if they are OS specific.

\chk{Model}:
%
Describe specific models (e.g., ImageNet, AlexNet, and MobileNets) used in your
thesis, their sizes, and how to access them.

\chk{Data set}:
%
Identify if your thesis uses any specific data sets, and briefly characterize
these data sets, and explain how to access them.

\chk{Run-time environment}:
%
Clarify whether your artifact is OS-specific (Linux, Windows, MacOS, Android,
etc.), and, if yes, list the version, (key) software dependencies (JIT, libs,
run-time adaptation frameworks, etc.), and whether the artifact requires
root (or administrator) privileges?

\chk{Hardware}:
%
Describe whether your thesis has specific hardware requirements (e.g.,
supercomputer, architecture simulator, GPU, and FPGA) or specific features
(e.g., hardware counters for measuring power consumption and root-level access
to CPU or GPU frequency).

\chk{Run-time state}:
%
Explain whether the evaluation of your artifact is sensitive to run-time state
(e.g., cold or hot cache, and network or cache contentions).

\chk{Execution}:
%
List any specific (environment) conditions (e.g., sole user, process pinning,
profiling, and adaptation) required by your artifact for any experiment, and
also how long it takes for the experiments (or executions) to run or finish.

\chk{Security, privacy, and ethical concerns}:
%
Highlight any specific security, privacy, or ethical concerns with running the
experiments (e.g., malware sample sandboxing and network scanning).

\chk{Metrics}:
%
Describe the metrics you report (e.g., execution time and inference per second)
in your experiments.

\chk{Output}:
%
Briefly explain the results generated in various experiments and their formats
(e.g., console, file, table, graph).
%
Indicate whether the expected results are included as part of the artifact(s).

\chk{Experiments}:
%
Explain how to prepare the environment for running experiments, replicate or
reproduce the results (for instance, explain the OS scripts to run, manual steps
to be taken by the user, etc.)?
%
We strongly recommend you to emphasize the \textit{maximum allowable variation}
in the (empirical) results.

\chk{Approximate disk space}:
%
Mention the storage requirements for the artifact and its experiments, so that
the people attempting to re-use or re-evaluate the artifact can provision their
environments appropriately.

\chk{Approximate preparation time}:
%
Emphasize how long it would take to prepare the environment and artifact before
anyone can run the experiments to reproduce the results.

\chk{Approximate time to complete experiments}:
%
Clarify how long it would take for a user to complete all experiments from start
to finish, excluding the preparation time.

\chk{Public availability}:
%
If your artifact is or will be publicly available, provide the reference or
location to access it, along with version and other metadata to fetch the exact
copy of the artifact used in this thesis.

\chk{Code license(s)}:
%
If your artifact will be publicly available, mention licenses, if any,
associated with one or more components of your artifact.

\chk{Data license(s)}:
%
If your artifact will be publicly available, mention licenses, if any, for the
data sets you used in your artifact(s).

\chk{Workflow framework(s)}:
%
Mention whether any workflow framework was (or could be) used for automating and
customizing the experiments.

\chk{Archive}:
%
If you are artifacts are archived on a publicly accessible portal, mention its
DOI, or stable reference.
%
If you archived the artifact(s) in a public versioning system (e.g., GitHub), which can evolve over
time, provide a (stable) reference (e.g., a URL pointing to the commit hash or
tag) to the version used in this thesis.


%%% Local Variables:
%%% mode: latex
%%% TeX-master: "../thesis"
%%% End:
